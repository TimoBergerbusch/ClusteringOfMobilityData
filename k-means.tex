%%This is a very basic article template.
%%There is just one section and two subsections.
\documentclass{article}
\usepackage[latin1]{inputenc} %coding of writteninput %latin1 allows for Umlaute
\usepackage[T1]{fontenc}%vectorized fonts (cm-super package)
\usepackage[german]{babel}%some specifics of the german language
\usepackage{amsfonts, amsmath, amsthm, amssymb, mathabx, paralist}
 \setlength{\parindent}{0em} 
  \usepackage{listings}
\usepackage{geometry}
  \geometry{a4paper, top=25mm, left=20mm, right=15mm, bottom=20mm, headsep=10mm, footskip=12mm}
  
 %Decisiontree
 \usepackage{tikz,forest}
\usetikzlibrary{arrows.meta}
  
\usepackage{graphicx} 

\usepackage{verbatim}%f�r txt datei

\usepackage{color} %red, green, blue, yellow, cyan, magenta, black, white
\definecolor{mygreen}{RGB}{28,172,0} % color values Red, Green, Blue
\definecolor{mylilas}{RGB}{170,55,241}

\begin{document}


\title{k-Means Algorithm}
\section{Algorithm in general}

https://www.datascience.com/blog/k-means-clustering

The critical steps are choosing a correct K and overfitting. We've got to find
the critical point, such that the improvement of a bigger K isn't that big
anymore. The ``elbow point''.

The second big point is finding a good starting set of centroids. It changes a
lot where you start. 


\end{document}
