%%This is a very basic article template.
%%There is just one section and two subsections.
\documentclass{article}
\usepackage[latin1]{inputenc} %coding of writteninput %latin1 allows for Umlaute
\usepackage[T1]{fontenc}%vectorized fonts (cm-super package)
\usepackage[german]{babel}%some specifics of the german language
\usepackage{amsfonts, amsmath, amsthm, amssymb, mathabx, paralist}
 \setlength{\parindent}{0em} 
  \usepackage{listings}
\usepackage{geometry}
  \geometry{a4paper, top=25mm, left=20mm, right=15mm, bottom=20mm, headsep=10mm, footskip=12mm}
 \usepackage{rotating} 
 %Decisiontree
 \usepackage{tikz,forest}
\usetikzlibrary{arrows.meta}
  
\usepackage{graphicx} 

\usepackage{verbatim}%f�r txt datei

\usepackage{color} %red, green, blue, yellow, cyan, magenta, black, white
\definecolor{mygreen}{RGB}{28,172,0} % color values Red, Green, Blue
\definecolor{mylilas}{RGB}{170,55,241}

\usepackage[section]{placeins}

\begin{document}


\title{Summary of Possibilities}
\section{Used Data}
N = Numerical

C = Categorical

HMV and FEV not in explanation. In data is no average time with \{AM,MD,PM,MN\}
foundable.

One row of data incomplete. \textbf{Note:} Found in line 126581: it contains a \texttt{\#N/A}-failure	
\begin{sidewaystable}					
\begin{tabular}{|c|c|c|c|c|c|}
In Dataset Name & Position &Variable				&Datatype		&Values				&Explanation\\\hline
ORIGEN	   		&1		   &Origin					&C				&[1, 413]			&Medell\'{i}n areas\\
DESTINO	   		&2	       &Destination				&C				&[1, 413]			&Medell\'{i}n
areas\\
MOTIVO			&3		   &Reason					&C				&[1, 7]				&Work, errands,\ldots\\
MODO			&4		   &Mean of transportation	&C				&[1, 7]				&Mean of
transportation\\ 
				&		   &Average time			&C				&{AM, MD, PM, MN}	&Average between
origin and departure times\\
RED				&6		   &Time interval			&C				&{AM, MD, PM, MN} 	&Interval of the
day\\
DURACION		&7		   &Duration 				&N				&					&Duration of the travel in
minutes\\ 	
DISTKM			&8		   &Distance 				&N 				&					&Distance between origin and
destination centroids of the areas\\
EDAD			&10		   &Age						&N				&					&Age\\
SEXO			&12		   &Gender					&C				&[1,2]				&Gender\\
EST				&9		   &Strata*					&C				&[1,6]				&Social strata\\
\end{tabular}
\end{sidewaystable}


\section{k-Means}

https://www.datascience.com/blog/k-means-clustering

The critical steps are choosing a correct K and overfitting. We've got to find
the critical point, such that the improvement of a bigger K isn't that big
anymore. The ``elbow point''.

The second big point is finding a good starting set of centroids. It changes a
lot where you start. 


\end{document}
