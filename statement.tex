\documentclass{article}
\usepackage[utf8]{inputenc}

\begin{document}
	
\section{Miriam}
My part of the practical part, the beginning of the practicum, was mostly the RapidMiner. I did the clustering and decision tree  mining with help of this tool. I evaluated the outcome and adapted the system, such that I tried different configurations and all sets generated. Also I gave input about the requirements of the testdata generator, especially the person vector. In the second step, together with Timo, we built the presentation. We merged the different parts and prepared the presentation. In the last step I firstly wrote down my own results and together with Timo joined all the different parts, corrected with Timo the resulting paper and we made the fine tuning of it at the end.

\section{Martin}
At beginning of the practicum I was mainly focused on experimenting with the Rapidminer toolkit to find ways to read, process and evaluate the raw data, especially using k-means without PCA (neglecting the given stratification) and using neural networks for classification using the given ground truth. I also was looking for clusters inside the whole dataset by generating different charts that visualize a subset of attributes of all data points with Rapidminer. This was part of the final presentation but was later cut out of the final paper. The testing of performance of the neural networks (which was labor intensive and error prone) was done with Timo to enable double checks, with whom I also wrote the corresponding part in the final paper. Additionally I wrote the abstract.

\section{Walter}

\section{Moritz}

\section{Timo}	
My main focus at the start of the practicum was to create a sophisticated generator for different testsets, with the sizes and properties like different sizes and distribution, for which I wrote the \texttt{testDataGenerator.py}. All this includes an analysis of the data with the whole team. Further, I implemented the vector computation of mapping movements to persons and computing the vector format proposed by Miriam. All those different options are choose-able through commandline flags. For the whole script I wrote comments and documentation in order to provide a reusable base.\\
Later through the project my focus was to aid during various tasks. Therefore, Martin and me build different neural net architectures and tested them under various conditions and wrote the results down for the final paper.\\
Also Miriam and I tried several different clusterings, which she build in RapidMiner. We also prepared and practiced the presentation, which included the tasks of thinking about topics, which were not included of the technical part of our team. Like, a precise but short introduction or reasons for the problem, that we cannot reproduce or find clusters, we faced.\\
After the presentation we merged the different written parts of our team members and made additions or replacements if necessary. After the feedback from María, also Miriam and I, restructured and basically (re-)wrote certain aspects. This 
was finished by proofreading and fine tuning through the two of us.
\end{document}