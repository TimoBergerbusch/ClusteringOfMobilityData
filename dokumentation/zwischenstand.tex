\documentclass[11pt]{article}
\usepackage{amsmath} % Math
\usepackage{amssymb} % Math symbols
\usepackage[english]{babel} % Language
\usepackage{fancyhdr} % Header
\usepackage[a4paper, total={15cm, 20cm}]{geometry} % Dimensions of the paper and the text area
\usepackage[utf8]{inputenc} % encoding in UTF, needed for umlauts if German
\usepackage{mathtools} % Text above arrows
\usepackage{msc} % Drawing MSCs
\usepackage{multicol} % Multiple columns
\usepackage[explicit]{titlesec} % Automatic section titles
\usepackage{tikz} % Diagrams
\usetikzlibrary{arrows.meta, automata, shapes, matrix}


% Other packages that might be useful in the future
\usepackage{mathptmx,amssymb,amsmath}
%\usepackage{lingmacros}
%\usepackage{tree-dvips}
%\usepackage{ulem}
%\usepackage{amsthm}
%\usepackage{amsbsy}
%\usepackage{textcomp,gensymb}
%\usepackage{graphicx}
%\usepackage{mathtools}

% Custom variant of msc environment:
% - No "msc" keyword, longer partial messages
% - Increased vertical distance between messages
% - Less distance to the frame left and right
% - Less distance between header and processes
% - Less distance between footer and frame
% - Passing all given options down to the msc environment
\newenvironment{cmsc}[1][]{\msc[msc keyword={}, self message width=1.1cm, level height=0.6cm, environment distance=1.2cm, head top distance=0.75cm, foot distance=0.5cm, #1]}{\endmsc}

% No indentation at new paragraphs
\setlength{\parindent}{0pt}

% Distance between columns
\setlength{\columnsep}{1cm}
% Vertical line between columns
\setlength{\columnseprule}{0.5pt}
\def\columnseprulecolor{\color{gray}}

% Settings
%\newcommand{\sheetNr}{1}

%% Header
\fancyhf{}
\pagestyle{fancy}
\lhead{Clustering of Mobility Data}
%\rhead{Timo Bergerbusch: 344408}
\setlength{\headheight}{28pt}

%% Automatic section titles
%\titleformat{\section}{\normalfont\Large\bfseries}{}{0em}{Exercise #1}
%\titleformat{\subsection}{\normalfont\large\bfseries}{}{0em}{#1)}

\begin{document}

\section*{Stand \today:}
\subsection*{Schlechte Daten}
\begin{itemize}
	\item keine ID's: kann keine zwei Movements von einer Person aggregieren
	\item[$\Rightarrow$] Einteilung in Gruppen nicht wirklich Personen zuzuordnen
	\item eine Person ist morgens in anderem Cluster wie abends
	\item[]
	\item Möglich wäre es Movements basierend auf: \texttt{Alter}, \texttt{Geschlecht}, \texttt{Strata} $\Rightarrow$ ID?
	\item Unterscheidung von zwei zeitgleichen Movements, welche demnach nicht der selben Person gehören können
	\item \underline{Problem}: false Positives: wenn zwei unterschiedliche Personen als eine erkannt werden
	\item Mögliche Lösung des Problems:
		\begin{itemize}
			\item (Möglicherweise als Datenbank-JOIN)
			\item als Join in Rapidminer
			\item als Erweiterung im Python Script
		\end{itemize}	
	\item ID als Penalty für ``schlechtes'' Clustering
	\item ID auffassen als Kategorie nicht als Zahlenwert
\end{itemize}

\subsection*{Deliverables}
\begin{itemize}
	\item ein Plotter für alle möglichen Variable-Kombinationen und auch so etwas wie: \texttt{DIST/TIME}
	\item Python Datensatz Verringerung
\end{itemize}

\end{document}
