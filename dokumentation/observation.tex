\section{Conclusion} \label{sec: observation}
	%	TODO (1 page)
	%	TODO: restructure. Have like 3 parts
	%	The greater the population, the more the borders between the stratas blur. 
	As an overall result we get, that we can not determine the strata based on the information we have.
	Using various datasets (c.f. \Cref{sec: proprocessing}), we witnessed, that having equally distributed datasets leads to an overall higher accuracy. This rules out the bias observed in the original data, where strata 2 and 3 are very dominant (c.f. \Cref{table: distribution normal}), but also reduces the set size from originally 124978 to $6*2038=12228$ entries equally distributed over all stratas.\\
	During clustering we observed, that using a different distance measure formula would not lead to a huge difference. Also, we detected that reducing the numbers of clusters leads to better results, but decreases the significance of the statement, that could be made.\\
	Using a different neural net architecture will most likely not chance the accuracy value drastically. As stated, more complex nets need more training data, which is restricted as mentioned above.\\
	What we can observe is, that using the stratified vectors, we are able to increase the performance and accuracy, but still have no sufficient predictions. Therefore we see, the more data we have, the higher is the variance within the stratas themselves. We cannot draw some kind of lines to distinguish between different stratas. Data points, which were outliers in the smaller sets, are now not considered to be outliers, because many others have the same characteristics. So the lines, we were able to draw for smaller datasets, are blurring.\\ %TODO: letzter satz doof	-> hihi	