	\section{Discussion} \label{sec: discussion}
%	\begin{itemize}
%		\item repr. working day
%		\item over/under std's
%		\item indiv. of lifestyle
%		\item inv. behaviour in same strata
%	\end{itemize}
	Regarding the results, of not predicting the strata and also not observing any meaningful clusters throughout the data in given and stratified form, we thought about various aspects having an impact on the movement. We state 4 main influencing aspects, explaining the variance throughout the data.
	
	\subsection{Representativity of the day}
	The day the data was collected on (c.f. \Cref{sec: introduction}) is not mentioned. Therefore we cannot know, if it is representative. The people, asked to enter their movement, could had an exceptional day, like a vacation day, a doctors appointment or a broken car and therefore behave different from a usual day. \\
	Also the day itself is not mentioned, so we cannot determine, if it was even a working day or a weekendday. This influences the behaviour drastically.
	
	\subsection{People living over/under standards}
	There are a lot of people spending more or less money than they actually have. So for example there are people not earning a lot of money, but still having a car. Or people earning a lot, but spend it just on holidays or save it for bad times.
	
%	In Liebe
%	deine Miriam

	\subsection{Individuality of lifestyle}
	Obviously, people are individual in their way of life. So there are people, with enough money to buy for example a car, but refuse to do so in order to reduce CO$_2$ emission, or simplydislike driving. On the other end of the spectrum, some people, who have little money, still own and drive a car daily in order to go to work, since they cannot afford an other apartment.\\
	One can think of multitude scenarios of people behaving different caused by their individuality.
	
	\subsection{Inverse behavior in a strata}
	As an example, take a look at the strata 6, which denotes the richest people considered. Their we have inter alia two groups: 
	\begin{enumerate}
		\setlength{\itemindent}{1cm}
		\item[1. Group]
		Hard working people, laboring 60+ hours per weak to earn their money. One can imagine that they have to move quite a lot since they are always busy.
		\item[2. Group]
		People, who are rich by birth, which do not have the necessity to work or move at all. They could possibly stay home and do not have to leave at all.
	\end{enumerate} 
	Those two groups are completely opposite, but still belong to the same strata. Now also considering strata 1, we can find the exact same movement behaviour in there as well. Within the poorest there are people that are wandering around the town via feet or bike, since they have no objective, like work. Also there are people, who do not move at all, because of the same reason.	
	
	So we can see that based on the information we have, it is very unlikely to have a clear correlation between the given data and the strata. However, given more information about the previously mentioned aspects and more entries in general, a kind of correlation could be found.\\
	Future research could include a component analysis of the decision tree and neural network in order to ascertain the influencing parts and therefore improve the data gathering. Also one could use different network visualization tools in order to infer patterns. An example would be to take the origin (destination) sectors as nodes, an edge, if there is a movement between, directed or undirected, and a different thickness or color gradient, based on the number of this edge being taken.\\
	
	Overall we can conclude that with the data we got, we are not able to find a correlation.