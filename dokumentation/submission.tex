% This is samplepaper.tex, a sample chapter demonstrating the
% LLNCS macro package for Springer Computer Science proceedings;
% Version 2.20 of 2017/10/04
%
\documentclass[runningheads]{llncs}
%
\usepackage{graphicx}
%\usepackage{ngerman}
\usepackage[utf8x]{inputenc}
\usepackage{fancyvrb}
\usepackage{courier}
\usepackage{helvet}
\usepackage{tikz}
\usepackage{xcolor}
\usepackage{pdfpages}
\usetikzlibrary{calc}
\usepackage[strict]{changepage}
\usepackage{xspace}
\usepackage{hyperref}
\usepackage{cleveref}
\usetikzlibrary{shapes.geometric}
\usetikzlibrary{arrows}
\usetikzlibrary{positioning}
\usetikzlibrary{arrows.meta, automata, shapes, matrix,positioning}
\usepackage{amssymb}
\usepackage{pifont}% http://ctan.org/pkg/pifont
\usepackage{subcaption} 
\usepackage{float}
\usepackage{fixfoot}
\usepackage{graphicx}
\usepackage{wrapfig}
\usepackage{multicol}
\usepackage{amsmath}
\usepackage{listings}

\newcommand{\cmark}{\ding{51}}%
\newcommand{\xmark}{\ding{55}}%
% Used for displaying a sample figure. If possible, figure files should
% be included in EPS format.
%
% If you use the hyperref package, please uncomment the following line
% to display URLs in blue roman font according to Springer's eBook style:
% \renewcommand\UrlFont{\color{blue}\rmfamily}

\begin{document}
	%
	\title{Clustering Analysis of Mobility Data}
	%
	%\titlerunning{Abbreviated paper title}
	% If the paper title is too long for the running head, you can set
	% an abbreviated paper title here
	%
	\author{Miriam Wagner\and
		Martin Breuer\and
		Moritz Werthebach\and
		Timo Bergerbusch\and
		Walter Schikowski}
	%
	\authorrunning{F. Author et al.} %TODO
	% First names are abbreviated in the running head.
	% If there are more than two authors, 'et al.' is used.
	%
	\institute{RWTH Aachen, Templergraben 55, 52062 Aachen, Germany}
	%
	\maketitle              % typeset the header of the contribution
	%
	\begin{abstract} %TODO
		The abstract should briefly summarize the contents of the paper in
		150--250 words.
		
		\keywords{Clustering \and Rapidminer \and Cluster \and Data Mining} %TODO: maybe more
	\end{abstract}
	%
	%
	%
	\section{Introduction}
	\section{Preprocessing}
	\begin{itemize}
		\item max equal of 2038 for a strata 
		\item max equal of 595 per persona in strata
		\item vector calc.
	\end{itemize}
	\subsection{Vector}
	Instead of simple IDs for every person we expand the parsing by using a data encapsulating in a class called \texttt{Person}. This class stores the ID, the parameters defining a person %TODO: ref zu code basics
	, and all movements from that person.\\
	Then we are able to compute the following vector, with 848 entries, for further usage, that combines all movements of the person:
	\begin{align*}
	\underbrace{\#o_1, \dots, \#o_{413}, \#d_1, \dots, \#d_{413}}_{2\cdot 413} ,
	\underbrace{\mathit{AM}, \mathit{MD}, \mathit{PM}, \mathit{MN}}_{4}, 
	\underbrace{\#r_1, \dots, \#r_7}_{7}, \\
	\underbrace{\#\mathit{MoT}_1, \dots, \#\mathit{MoT}_7}_{7}, \underbrace{\mathit{SDest}, \mathit{SDist}, \mathit{G}, \mathit{A} ,\mathit{strata}, \mathit{strataGrouped}}_{6}
	\end{align*}
	with the following abbreviations ($1 \le i \le 413$, $1 \le j \le 7$):
	\begin{multicols}{2}
		\begin{itemize}
			\setlength{\itemindent}{.4cm}
			\item[$o_i$:]  the $i$-th origin data point
			\item[$d_i$:]  the $i$-th destination data point
			\item[$\mathit{AM}$:] movements at time stamp AM
			\item[$\mathit{MD}$:] movements at time stamp MD
			\item[$\mathit{PM}$:] movements at time stamp PM
			\item[$\mathit{MN}$:] movements at time stamp MN
			\item[$r_j$:] the $j$-th reason
			\item[$\mathit{MoT}_j$:] the $j$-th mean of transportation
			\item[$\mathit{SDest}$:] sum of all durations
			\item[$\mathit{SDist}$:] sum of all distances
			\item[$\mathit{G}$:] the gender
			\item[$\mathit{A}$:] the age
			\item[$strata$:] the strata (used for comparison)
			\item[$strataGrouped$:] the aggregated stratas
		\end{itemize}
	\end{multicols}
	\section{Predicting}
	
	\subsection{Classification}
	\subsection{Neural Net}
	Also we have a max. of 595 persons for strata 6 to have an equally distributed test set.\\	
	All using the vector data\\
	
	We consider 3 neural nets $\mathcal{N}_1,\mathcal{N}_2$ and $\mathcal{N}_3$, all having 4 hidden layers, 50 epochs and 10 iterations.	
	As an example of other strata aggregation we combine the stratas 1--2, 3--4 and 5--6 together and call them $\mathcal{N}_i^\star$, for $i \in \{5,10,20\}$. For that we take an equal distribution of all original stratas and map them correspondingly to the new stratas. 	
	\setlength\tabcolsep{.2cm}
	\begin{figure}[H]
	\centering
	\begin{tabular}{|c|c|c|c|c|c|}
		\hline
		                         &   \#    &        & \multicolumn{3}{c|}{Set size} \\
		          Name           & Neurons &   AG   &  100  &  200  &      595      \\ \hline
		    $\mathcal{N}_5$      &    5    & \xmark & 60.03 & 59.92 &     60.18     \\
		 $\mathcal{N}_5^\star$   &    5    & \cmark & 87.6  & 89.7  &     71.05     \\
		   $\mathcal{N}_{10}$    &   10    & \xmark & 75.83 & 73.54 &     69.56     \\
		$\mathcal{N}_{10}^\star$ &   10    & \cmark & 92.93 & 93.48 &     74.58     \\		
		%		$\mathcal{N}_{10}^\star$ &   10    & \cmark & 88.33 {\small $\pm$7.49} & 90.67{\small $\pm$2.81} & 92.14 {\small $\pm$ 2.59} \\
		   $\mathcal{N}_{20}$    &   20    & \xmark & 75.45 & 71.14 &     61.87     \\
		$\mathcal{N}_{20}^\star$ &   20    & \cmark & 92.87 & 94.4  &     78.32     \\ \hline
	\end{tabular}
	\caption{The accuracy values of the neural nets. (See excel-spreadsheet)}
	\label{tab: nn-accuracy}
	% mittels deep-learning-vector-average
	\end{figure}


If we take 50 neurons per layer we have $14 \cdot 50^4 \cdot 6 \approxeq 525.000.000$ synapses for which the input data set would be to small to have sufficient training.\\

The greater the population, the more the borders between the stratas blur. 

	
	
\end{document}
