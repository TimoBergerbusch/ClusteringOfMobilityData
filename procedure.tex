%%This is a very basic article template.
%%There is just one section and two subsections.
\documentclass{article}
\usepackage[latin1]{inputenc} %coding of writteninput %latin1 allows for Umlaute
\usepackage[T1]{fontenc}%vectorized fonts (cm-super package)
\usepackage[german]{babel}%some specifics of the german language
\usepackage{amsfonts, amsmath, amsthm, amssymb, mathabx, paralist}
 \setlength{\parindent}{0em} 
  \usepackage{listings}
\usepackage{geometry}
  \geometry{a4paper, top=25mm, left=20mm, right=15mm, bottom=20mm, headsep=10mm, footskip=12mm}
 \usepackage{rotating} 
 %Decisiontree
 \usepackage{tikz,forest}
\usetikzlibrary{arrows.meta,arrows,shadows,positioning,decorations.pathreplacing}
\pgfdeclarelayer{background}
\pgfdeclarelayer{foreground}
\pgfsetlayers{background,main,foreground}
\usepackage{float}
\usepackage{enumitem}



  
\usepackage{graphicx} 

\usepackage{verbatim}%f�r txt datei

\usepackage{color} %red, green, blue, yellow, cyan, magenta, black, white
\definecolor{mygreen}{RGB}{28,172,0} % color values Red, Green, Blue
\definecolor{mylilas}{RGB}{170,55,241}

\usepackage[section]{placeins}

\begin{document}


\title{General Procedure}

\section{General Procedure}

	% Define block styles
	\tikzstyle{materia}=[draw, fill=blue!20, text width=6.0em, text centered, minimum height=1.5em,drop shadow]
	\tikzstyle{etape} = [materia, text width=8em, minimum width=10em, minimum height=3em, rounded corners, drop shadow]
	\tikzstyle{texto} = [above, text width=6em, text centered]
	\tikzstyle{linepart} = [draw, thick, color=black!50, -latex', dashed]
	\tikzstyle{line} = [draw, thick, color=black!50, -latex']
	\tikzstyle{ur}=[draw, text centered, minimum height=0.01em]
	
	% Define distances for bordering
	\newcommand{\blockdist}{1.3}
	\newcommand{\edgedist}{1.5}
	
	\newcommand{\etape}[2]{node (p#1) [etape]
		{#2}}
	
	% Draw background
	\newcommand{\background}[5]{%
		\begin{pgfonlayer}{background}
			% Left-top corner of the background rectangle
			\path (#1.west |- #2.north)+(-0.5,0.25) node (a1) {};
			% Right-bottom corner of the background rectanle
			\path (#3.east |- #4.south)+(+0.5,-0.25) node (a2) {};
			% Draw the background
			\path[fill=yellow!20,rounded corners, draw=black!50, dashed]
			(a1) rectangle (a2);
			\path (#3.east |- #2.north)+(0,0.25)--(#1.west |- #2.north) node[midway] (#5-n) {};
			\path (#3.east |- #2.south)+(0,-0.35)--(#1.west |- #2.south) node[midway] (#5-s) {};
			\path (#3.east |- #2.north)+(0.7,0)--(#3.east |- #4.south) node[midway] (#5-w) {};
	\end{pgfonlayer}}
	
	\newcommand{\transreceptor}[3]{%
		\path [linepart] (#1.east) -- node [above]
		{\scriptsize #2} (#3);}

	The setup I suggest within this project is the following:
	\begin{figure}[H]
		\centering
		\begin{tikzpicture}[scale=0.7,transform shape]
		
		% Draw diagram elements
		\path \etape{1}{raw Data};
		
		\path (p1.south)+(0.0,-1.5) \etape{2}{new Data};
		
		\path (p2.south)+(-3.0,-2) \etape{4}{PCA};
		\path (p2.south)+(3.0,-2) \etape{5}{Decision Tree};
%		\node [below=of p2] (p4-5) {};
		
		\path (p2.south)+(0.0,-4.0) \etape{6}{thin Data};
%		\node [below=of p2] (p6) {};
		
		\path (p6.south)+(-4.0,-2.0) \etape{7}{k-modes};
		\path (p6.south)+(0.0,-2.0) \etape{8}{rapidminer};
		\path (p6.south)+(4.0,-2.0) \etape{9}{PAM};
		\node [below=of p6] (p7-9) {};
		
		
		\path (p8.south)+(0.0,-2.0) \etape{10}{new stratas};
%		\path (p7.south)+(0.0,-2.0) \etape{9}{abnormal};
%		\node [below=of p8] (p10) {};
		
		% Draw arrows between elements
		\path [line] (p1.south) -- node [above] {} (p2);
		\path [line] (p2.south) -- node [above] {} (p4);
		\path [line] (p2.south) -- node [above] {} (p5);
		\path [line] (p4.south) -- node [above] {} (p6);
		\path [line] (p5.south) -- node [above] {} (p6);
		\path [line] (p6.south) -- node [above] {} (p7);
		\path [line] (p6.south) -- node [above] {} (p8);
		\path [line] (p6.south) -- node [above] {} (p9);
		\path [line] (p7.south) -- node [above] {} (p10);
		\path [line] (p8.south) -- node [above] {} (p10);
		\path [line] (p9.south) -- node [above] {} (p10);
		
		\background{p1}{p1}{p2}{p2}{bk1}
		\background{p4}{p4}{p5}{p5}{bk2}
		\background{p7}{p7}{p9}{p9}{bk3}
%		\background{p4}{p5}{p10}{p10}{bk4}
		
%		\path [line] (p5.south) -- node [above] {} (bk3-n);
%		\path [line] (bk3-s) -- node [above] {} (p8);
%		\path [line] (bk3-s) -- node [above] {} (p9);
		\path (bk1-w)+(+9.0,0) node (ur1)[ur] {};
		\path (bk2-w)+(+6.0,0) node (ur2)[ur] {};
		\path (bk3-w)+(+5.0,0) node (ur3)[ur] {};
		\transreceptor{bk1-w}{pre processing}{ur1};
		\transreceptor{bk2-w}{variable elimination}{ur2};
		\transreceptor{bk3-w}{clustering}{ur3};
		
		\draw [decorate,decoration={brace,amplitude=10pt,raise=5pt},yshift=0pt]
		(ur2)+(0,1) -- (11,-11) node [black,midway,xshift=1.5cm] {Pipeline};
		\end{tikzpicture}
		\caption{The clustering pipeline}
		\label{fig:pipeline}
	\end{figure}
	
	The suggested clustering pipeline is structured in 2 basic layers, where the second layer is sub structured in 2 separate parts:
	\begin{enumerate}
		\item \textbf{the ``pre processing''-Layer}:\\
			dividing the dataset into smaller datasets or change to equal distribution using for example ``Multivariate kernel density estimation''. Also we are able to neglect the strata.
		\item \textbf{``clustering pipeline''-Layer}:\\
			This Layer can be split into two parts and performs the steps on the input data, which can be changed 
			\begin{enumerate}[label=2.\arabic*.]
				\item \textbf{``variable elimination''-Layer}:\\
					compute neglect able variables, for example gender, using \texttt{PCA} or Decision Trees
				\item \textbf{``thin Data''-Layer}:\\
					remask the data by eliminating the variables proven to be insignificant
				\item \textbf{``clustering''-Layer}:\\
					cluster the thinned data using for example \texttt{k-modes}, \texttt{rapidminer} or \texttt{PAM}
				\item \textbf{new stratas-Layer}:
					the newly clustered stratas can now be compared to given stratas and visualized and interpreted
			\end{enumerate}				
	\end{enumerate}
\end{document}
